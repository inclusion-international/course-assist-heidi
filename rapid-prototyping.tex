\documentclass[aspectratio=169]{beamer}
\usetheme[style=noir]{fhtw}
\usepackage{preamble}

\title[Rapid Prototyping]{Rapid Prototyping}
\subtitle{CI - ASSIST HEIDI}
\author{Alija Sabic, \glsentrytext{msc}}
\mail{sabic.alija@gmail.com}
\institute{Department Electronic Engineering}

\NewDocumentCommand{\res}{ m }{``\usebibentry{#1}{title}'', \usebibentry{#1}{url}, \cite{#1}}

% \setlength{\parindent}{1em}
\setlength{\parskip}{1em}

\begin{document}

\begin{frame}[plain]
    \titlepage
\end{frame}

% \section{Introduction}
\section{Literature}
\begin{frame}
    \begin{itemize}
        \item \res{fasterman:2012}
        \item \res{benchoff:2014:1}
        \item \res{benchoff:2014:2}
        \item \res{hubs}
    \end{itemize}
\end{frame}

\section{Rapid Prototyping}

\begin{frame}
    \par \textit{Generative/additive manufacturing processes}.
    The object is formed directly from shapeless or shape-neutral starting materials by physical or chemical processes directly form the object.
    In contrast, there is \textit{subtractive manufacturing processes}, for instance.
    \par Classification also according to use:
    \begin{description}
        \item[Rapid Prototyping] the product is used as a prototype.
        \item[Rapid Tooling] the product is used as a tool.
        \item[Rapid Manufacturing] the product is used directly.
    \end{description}
\end{frame}

\section{Technologies}

\begin{frame}

\end{frame}

\subsection{\glsentrydesc{fdm}}
\begin{frame}
    \begin{itemize}
        \item \gls{fdm} \texttrademark
        \item \gls{fff} (free term)
    \end{itemize}
\end{frame}

\subsection{FabLabs}
\begin{frame}
    \par Fabrication Laboratory
    \begin{itemize}
        \item Open workplaces
        \item Typical machines
              \begin{itemize}
                  \item 3D printers
                  \item Laser cutters
                  \item \acs{cnc} machines
                  \item Presses
                  \item \ldots
              \end{itemize}
        \item Vienna: HappyLab or MetaLab
    \end{itemize}
\end{frame}

\begin{frame}{Social Impact}
    \begin{block}{The Economist, A third industrial revolution}
        ``As manufacturing goes digital, a third great change is now gathering space.
        It will allow things to be made economically in much smaller numbers, more flexibly and with a much lower input of labor.''
    \end{block}
    \begin{itemize}
        \item Decentralization of production (shift towards consumers)
        \item Sustainability and democratization $\rightarrow$ maker movement
        \item Discussion about patents and licensing
    \end{itemize}
\end{frame}

\section{3D Printing}
\subsection{\glsentrydesc{fdm}}
\begin{frame}
    \begin{itemize}
        \item Material (filament) is applied layer by layer
        \item Movable print head
        \item Precision down to \SI{0.1}{\milli\meter} possible (typical: \SIrange{0.3}{0.5}{\milli\meter})
        \item Speed and precision depends on diameter (typical \SI{0.4}{\milli\meter})
        \item Material: \SI{1.75}{\milli\meter} or \SI{3}{\milli\meter} diameter
    \end{itemize}
\end{frame}

\begin{frame}

\end{frame}

\subsection{\glsentrydesc{dlp}}
\begin{frame}
    \begin{itemize}
        \item Liquid photopolymers (photosensitive)
        \item Printed object is pulled out of the material from bottom to top
        \item Each layer is partially cured by \acs{uv} light projector
        \item Higher precision than with \acs{fff} is possible
        \item Prices are also decreasing (inexpensive printers $\approx$ \SI{250}{\sieuro})
    \end{itemize}
\end{frame}

\begin{frame}

\end{frame}

\subsection{\glsentrydesc{sls}}

\begin{frame}
    \begin{itemize}
        \item Laser beam heats powdery material
        \item Melting or sintering
        \item Expensive to purchase
        \item High powder wear
    \end{itemize}
\end{frame}

\subsection{Materials}
\begin{frame}
    \begin{itemize}
        \item \acs{pla}
              \begin{itemize}
                  \item Polyactide, biocompatible/biodegradable
                  \item Printing temperature: $\approx$ \SI{200}{\celsius}
                  \item Easiest material for \acs{fff} printing
              \end{itemize}
        \item Biocompounds
              \begin{itemize}
                  \item Plastics produced from biological feedstock
                  \item Various properties, e.g. GreenTec from Extrudr
                        \begin{itemize}
                            \item Biodegradable
                            \item Food safe
                            \item Dimensionally stable up to \SI{120}{\celsius}
                        \end{itemize}
              \end{itemize}
    \end{itemize}
\end{frame}

\begin{frame}
    \begin{itemize}
        \item \acs{abs}
              \begin{itemize}
                  \item \acl{abs} copolymers
                  \item Recyclable (non-biodegradable)
                  \item Printing temperature: $\approx$ \SI{220}{\celsius}
                  \item Somewhat difficult material for \acs{fff} printing
                  \item Very resilient
              \end{itemize}
        \item \acs{petg}
              \begin{itemize}
                  \item \acl{petg}
                  \item Recyclable (non-biodegradable)
                  \item Printing temperature: $\approx$ \SI{250}{\celsius}
                  \item Food safe
                  \item Easy to use
                  \item Not all printers can work without problems at this high temperature
              \end{itemize}
    \end{itemize}
\end{frame}

\begin{frame}
    \begin{itemize}
        \item \acs{pva}
              \begin{itemize}
                  \item \acl{pva}
                  \item Water-soluble
                  \item Printing temperature: $\approx$ \SI{200}{\celsius}
                  \item Used as support material and dissolved in water after printing
              \end{itemize}
        \item Photopolymer
              \begin{itemize}
                  \item Used in \acs{dlp} printers
                  \item Composition depends on manufacturer
              \end{itemize}
    \end{itemize}
\end{frame}

\subsection{Workflow}
\begin{frame}
    \begin{enumerate}
        \item Plan/model the model (or use existing model)
        \item Export the model in \acs{stl} format (\acs{cad} software)
        \item Convert \acs{stl} file into printer compatible layer model (slicer)
        \item Export layer model to G-code (for \acs{fff}, but not for \acs{dlp})
        \item Print
    \end{enumerate}
\end{frame}

\subsection{\glsentrytext{stl} Files}
\begin{frame}
    \begin{itemize}
        \item \acl{stl}
        \item Description by single triangle facets
        \item Generates only an approximation
        \item Especially difficult: curves/round bodies
    \end{itemize}
    \begin{exampleblock}{}
        Generated from 3D modeling software or downloaded from the internet.
        Platform compatible, can be used for any generative manufacturing process.
    \end{exampleblock}
\end{frame}

\subsection{Slicing}
\begin{frame}
    \begin{itemize}
        \item Creates a layer model from \acs{stl} files
        \item Every movement of the print head is calculated (only for \acs{fff})
        \item Many adjustment possibilities
        \item Accuracy/speed/material of print if defined here
    \end{itemize}
    \begin{exampleblock}{}
        The slicer generates G-code from the \acs{stl} file.
        This G-code depends on: printer, material, speed, accuracy, etc.
        In \acs{dlp} printing, a projected image is created for each layer.
    \end{exampleblock}
\end{frame}

\subsection{G-code}
\begin{frame}
    \begin{itemize}
        \item \acs{rs}-274, originally for \acs{cnc} machines (only for \acs{fff})
        \item Pure text description of the execution of the commands
        \item G-code is sent to the 3D printer either via an external medium (\acs{usb} stick, \acs{sd} card) or via cable
        \item \texttt{G} \ldots\, movement of the tool/print head
        \item \texttt{M} \ldots\, miscellaneous functions
        \item \texttt{F} \ldots\, feed
        \item \texttt{X/Y/Z} \ldots\, absolute or relative x, y, z coordinates
    \end{itemize}
    \begin{exampleblock}{}
        G-code expresses the exact instructions for the 3D printer.
        The commands are printer and material specific.
    \end{exampleblock}
\end{frame}

\begin{frame}
    \begin{listing}[H]
        \inputsource[]{gcode}{gcode/example.gcode}
        \caption{G-code example}
        \label{lst:gcode:example}
    \end{listing}
\end{frame}

\section{Software and Platforms}

\subsection{Thingiverse}
\begin{frame}
    \begin{itemize}
        \item Platform for the exchange of 3D models
        \item Extensive collection
        \item Many \acs{at} models
        \item Attention: not all models are suitable for \acs{fff}
    \end{itemize}
\end{frame}

\subsection{\glsentrytext{cad}}
\subsubsection{Commercial}
\begin{frame}
    \begin{itemize}
        \item SolidWorks
        \item Auto\acs{cad}
        \item \acs{catia}
        \item SketchUp
        \item Inventor
        \item Fusion360
        \item SolidEdge
    \end{itemize}
\end{frame}

\subsubsection{Free}
\begin{frame}
    \begin{itemize}
        \item Free\acs{cad}
        \item Blender
        \item AutoDesk Inventor
        \item OnShape
        \item OpenSCAD
        \item Libre\acs{cad}
        \item 123D (discontinued)
    \end{itemize}
\end{frame}

\section{Laser Cutter}

\subsection{Technology}

\begin{frame}
    \begin{itemize}
        \item Subtractive cutting/engraving by concentrated radiation
        \item Ablation: material removal by heating up
        \item Absorption of energy in the material
        \item Continuous or pulsed laser beams
        \item Almost all materials can be processed
    \end{itemize}
\end{frame}

\subsection{Laser}

\begin{frame}
    \begin{itemize}
        \item Electromagnetic waves
        \item High intensity
        \item Monochromatic wavelength
        \item Sharp focussing
        \item For laser cutting: infrared light
    \end{itemize}
\end{frame}

\subsection{Advantages}

\begin{frame}
    \begin{itemize}
        \item Complex outlines possible
        \item Precise
        \item Fast
        \item Low minimum number of copies
        \item Cutting/engraving in one step possible
    \end{itemize}
\end{frame}

\subsection{Disadvantages}

\begin{frame}
    \begin{itemize}
        \item High acquisition cost
        \item Hazard protection
        \item Low electrical efficiency ($\approx$ \SI{20}{\percent})
        \item Repeatability (especially for low cost machines)
        \item Many, but not all materials possible
    \end{itemize}
\end{frame}

\subsection{Products}

\begin{frame}
\end{frame}

\begin{frame}
\end{frame}

\subsection{Examples}

\begin{frame}
\end{frame}

\begin{frame}
\end{frame}

\section{Other Technologies}

\subsection{Lego \texttrademark}

\begin{frame}
    \begin{itemize}
        \item Very fast
        \item Modular
        \item Many different elements
        \item Combination with other technologies
    \end{itemize}
\end{frame}

\subsection{Plaast/Instamorph}

\begin{frame}
    \begin{itemize}
        \item Heatable polymer
        \item Becomes soft with hot water
        \item Hard at room temperature
    \end{itemize}
\end{frame}

\begin{frame}
\end{frame}

\subsection{\glsentrytext{pvc} Pipes}

\begin{frame}
    \begin{itemize}
        \item Very cheap
        \item Easy to obtain
        \item Does not require special tools/machines
        \item Therese Willkomm (``\acs{at} Solutions in Minutes'')
    \end{itemize}
\end{frame}

\begin{frame}
\end{frame}

\subsection{LocLine}

\begin{frame}
    \begin{itemize}
        \item Cheap
        \item Easy to obtain
        \item Does not require special tools/machines
        \item From \acs{cnc} industry (coolant hose)
    \end{itemize}
\end{frame}

\appendix

\begin{frame}[allowframebreaks]{Acronyms}
    \glsadd{pvc}
    \glsadd{sls}
    \printglossary[type=\acronymtype, nonumberlist]
\end{frame}

\begin{frame}[allowframebreaks]{References}
    \bibliography{references}
\end{frame}

\end{document}