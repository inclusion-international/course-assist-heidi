\documentclass[aspectratio=169]{beamer}
\usetheme[style=noir]{fhtw}
\usepackage{preamble}
\resetcounteronoverlays{listing}

\title[TensorFlow Lite]{TensorFlow Lite}
\subtitle{\glsentrytext{ci} - ASSIST \glsentrytext{heidi}}
\author{Alija Sabic, \glsentrytext{msc}}
\mail{sabic@technikum-wien.at}
\institute{Department Electronic Engineering}

\begin{document}

\begin{frame}[plain]
    \titlepage
\end{frame}

\section{TensorFlow}
\begin{frame}
    \par \href{https://www.tensorflow.org/}{TensorFlow} is a powerful open-source machine learning framework developed by Google.
    It is widely used in various fields, including computer vision, natural language processing, and deep learning.
    TensorFlow provides a platform for building and training machine learning models that can be deployed in various applications.
    \par \href{https://www.tensorflow.org/lite}{TensorFlow Lite} is a lightweight version of TensorFlow designed for mobile and embedded devices.
    It enables developers to run machine learning models on devices with limited resources, such as smartphones, tablets, and IoT devices.
    TensorFlow Lite offers optimized performance, smaller model sizes, and lower power consumption compared to the standard version of TensorFlow.
\end{frame}

\section{Introduction}
\begin{frame}
    \begin{figure}
        \includegraphics[width=0.85\textwidth]{images/tflite/sinewave-model-overview.jpg}
        \caption{\glsentrydesc{ml} model workflow.}
    \end{figure}
\end{frame}

\section{Model}
\subsection{Design}
\begin{frame}
    \begin{listing}[H]
        \inputsource[fontsize=\fontsize{10}{10}, firstline=50, lastline=63]{c}{arduino/sinewave-test.c}
        \caption{Project code: load the model.}
        \label{lst:arduino:tflite:sinewave:setup:model}
    \end{listing}
\end{frame}
\subsection{Google Colab}
\begin{frame}
    \par Open \href{https://colab.research.google.com/}{Google Colab} and create a new notebook.
    \begin{listing}[H]
        \inputsource[firstline=10, lastline=14]{py}{tflite/tflite_sinewave_training.py}
        \caption{IPython magic.}
        \label{lst:tflite:sinewave:magic}
    \end{listing}
\end{frame}
\subsection{Import}
\begin{frame}
    \begin{listing}[H]
        \inputsource[firstline=16, lastline=20]{py}{tflite/tflite_sinewave_training.py}
        \caption{Import dependencies.}
        \label{lst:tflite:sinewave:import}
    \end{listing}
    \begin{listing}[H]
        \inputsource[firstline=22, lastline=26]{py}{tflite/tflite_sinewave_training.py}
        \caption{Print version information.}
        \label{lst:tflite:sinewave:version}
    \end{listing}
\end{frame}
\subsection{Settings}
\begin{frame}
    \begin{listing}[H]
        \inputsource[fontsize=\fontsize{10}{10}, firstline=28, lastline=39]{py}{tflite/tflite_sinewave_training.py}
        \caption{Define settings.}
        \label{lst:tflite:sinewave:settings}
    \end{listing}
\end{frame}
\subsection{Samples}
\begin{frame}
    \begin{listing}[H]
        \inputsource[fontsize=\fontsize{9}{9}, firstline=41, lastline=64]{py}{tflite/tflite_sinewave_training.py}
        \caption{Generate random samples.}
        \label{lst:tflite:sinewave:generate_samples}
    \end{listing}
\end{frame}

\begin{frame}
    \begin{figure}
        \includegraphics[width=0.85\textwidth]{images/tflite/colab/x-values.png}
        \caption{Random samples (x values).}
    \end{figure}
\end{frame}

\begin{frame}
    \begin{figure}
        \includegraphics[width=0.85\textwidth]{images/tflite/colab/y-values.png}
        \caption{Random samples (y values).}
    \end{figure}
\end{frame}

\begin{frame}
    \begin{figure}
        \includegraphics[width=0.85\textwidth]{images/tflite/colab/sets.png}
        \caption{Random sample sets: training, validation, test.}
    \end{figure}
\end{frame}
\subsection{Model}
\begin{frame}
    \begin{listing}[H]
        \inputsource[fontsize=\fontsize{10}{10}, firstline=50, lastline=63]{c}{arduino/sinewave-test.c}
        \caption{Project code: load the model.}
        \label{lst:arduino:tflite:sinewave:setup:model}
    \end{listing}
\end{frame}
\subsection{Train}
\begin{frame}
    \begin{listing}[H]
        \inputsource[fontsize=\fontsize{10}{10}, firstline=78, lastline=95]{py}{tflite/tflite_sinewave_training.py}
        \caption{Train the model.}
        \label{lst:tflite:sinewave:train}
    \end{listing}
\end{frame}

\begin{frame}
    \begin{figure}
        \includegraphics[width=0.85\textwidth]{images/tflite/colab/history.png}
        \caption{Training and validation loss.}
    \end{figure}
\end{frame}
\subsection{Test}
\input{ml/tf/lite/sinewave/colab/test}
\subsection{Convert}
\begin{frame}
    \par The TensorFlow Lite Model file can be inspected with programs such as \href{https://github.com/lutzroeder/netron}{Netron} (\href{https://netron.app/}{Netron Browser App}).
    \begin{listing}[H]
        \inputsource[fontsize=\fontsize{9}{10}, firstline=107, lastline=112]{py}{tflite/tflite_sinewave_training.py}
        \caption{Convert to TensorFlow Lite model.}
        \label{lst:tflite:sinewave:tflite_model}
    \end{listing}
    \begin{listing}[H]
        \inputsource[fontsize=\fontsize{9}{10}, firstline=142]{py}{tflite/tflite_sinewave_training.py}
        \caption{Convert to C source file.}
        \label{lst:tflite:sinewave:tflite_model:c}
    \end{listing}
\end{frame}

\begin{frame}
    \begin{listing}[H]
        \inputsource[fontsize=\fontsize{9}{10}, firstline=114, lastline=140]{py}{tflite/tflite_sinewave_training.py}
        \caption{Converter function.}
        \label{lst:tflite:sinewave:tflite_model:c:converter}
    \end{listing}
\end{frame}
\subsection{Resources}
\begin{frame}
    \begin{itemize}
        \item ``Deep Learning'' by Ian Goodfellow, Yoshua Bengio, Aaron Courville, 2016 (\url{https://www.deeplearningbook.org/})
        \item ``Introduction to Machine Learning and Deep Learning: A Hands-On Starter's Guide'' by Samuel Dodge, Lina Karam, 2019 (\url{https://www.deeplearningtextbook.org})
        \item ``\acs{ai} for Everyone'' by Andrew Ng, 2022 (\url{https://www.coursera.org/learn/ai-for-everyone})
        \item ``TinyML - Machine Learning with TensorFlow on Arduino'' by Pete Warden, Daniel Situnayake, 2019 (\url{https://www.oreilly.com/library/view/tinyml/9781492052036/})
    \end{itemize}
\end{frame}

\section{Inference}
\subsection{Installation}
\begin{frame}[fragile]
    \par The officially supported TensorFlow Lite Micro library for Arduino\textregistered{} resides in the \href{https://github.com/tensorflow/tflite-micro-arduino-examples}{tflite-micro-arduino-examples} GitHub repository.
    \par To install the in-development version of this library, you can use the latest version directly from the GitHub repository.
    This requires you clone the repo into the folder that holds libraries for the Arduino\textregistered{} \acs{ide}.
    \par The location for this folder varies by operating system, but typically it's in \texttt{\textasciitilde{}/Arduino/libraries} (Linux), \texttt{\textasciitilde{}/Documents/Arduino/libraries/} (Mac\acs{os}), and \texttt{My Documents\textbackslash{}Arduino\textbackslash{}Libraries} (Windows).
    \begin{listing}[H]
        \begin{mdframed}
            \begin{minted}[autogobble, fontsize=\fontsize{7}{7}, linenos=false]{sh}
                git clone https://github.com/tensorflow/tflite-micro-arduino-examples Arduino_TensorFlowLite
            \end{minted}
        \end{mdframed}
        \caption{Download TensorFlow Lite.}
        \label{lst:tflite:installation}
    \end{listing}
\end{frame}
\subsection{Project}
\begin{frame}[fragile]
    \par Create a new project using the Arduino\textregistered{} \acs{ide} and copy the file (\texttt{sine\_model.h}, downloaded from Google) Colab to your project folder.
    \begin{listing}[H]
        \inputsource[fontsize=\fontsize{8}{8}, lastline=7]{c}{arduino/sine_model.h}[hidealllines=true, tikzsetting={draw=black, line width=0.35pt}, leftline=true, rightline=true, topline=true]
        \vspace{-1.25em}
        \begin{mdframed}[hidealllines=true, tikzsetting={draw=black, line width=0.35pt}, leftline=true, rightline=true]
            \begin{minted}[gobble=14, fontsize=\fontsize{8}{8}, linenos=false]{c}
                ...
            \end{minted}
        \end{mdframed}
        \vspace{-1.25em}
        \inputsource[fontsize=\fontsize{8}{8}, firstline=268]{c}{arduino/sine_model.h}[hidealllines=true, tikzsetting={draw=black, line width=0.35pt}, leftline=true, rightline=true, bottomline=true]
        \caption{Created model as C array (\texttt{sine\_model.h}).}
        \label{lst:tflite:sinewave:model:c_array}
    \end{listing}
    \par Alternatively, you can create a \texttt{C/C++} source file using \texttt{xxd} (Linux).
    \begin{listing}[H]
        \begin{mdframed}
            \begin{minted}[autogobble, linenos=false]{sh}
                xxd -i sine_model.tflite > sine_model.cc
            \end{minted}
        \end{mdframed}
        \caption{Download TensorFlow Lite.}
        \label{lst:tflite:sinewave:model:conversion}
    \end{listing}
\end{frame}
\subsection{Circuit}
\begin{frame}
    \begin{figure}
        \begin{tikzpicture}
            \ctikzset{bipoles/length=1cm, !vi/.style={no v symbols, no i symbols}, bipole voltage style/.style={text opacity=0}, bipole current style/.style={color=ttw-red}}
            \draw (2,4) node[rp2040] (rp20401) {}
            (rp20401.D2) to ++(1,0) to ++(0,4) to ++(2,0) to ++(0,-1) to [R,name=R,l={$R_1 = \SI{220}{\ohm}$},v=$U_1$,voltage shift=3.5,!vi] ++(0,-1) to [leDo,name=LED,l=$D_1$,v=$U_2$,i=$i$,voltage shift=3.5,!vi] ++(0,-2)
            (rp20401.GNDR) to ++(3,0) to ++(0,1);
            \fixedvlen[0.5cm]{R}{$U_1$}[tw-blue]
            \fixedvlen[0.5cm]{LED}{$U_2$}[tw-blue]
            \iarronly{LED}
        \end{tikzpicture}
        \caption{Project Circuit.}
    \end{figure}
\end{frame}
\subsection{Imports}
\begin{frame}
    \begin{listing}[H]
        \inputsource[fontsize=\fontsize{9}{9}, lastline=11]{c}{arduino/sinewave-test.c}
        \caption{Project code: import dependencies.}
        \label{lst:arduino:tflite:sinewave:import_define}
    \end{listing}
\end{frame}
\subsection{Settings}
\begin{frame}
    \begin{listing}[H]
        \inputsource[fontsize=\fontsize{10}{10}, firstline=28, lastline=39]{py}{tflite/tflite_sinewave_training.py}
        \caption{Define settings.}
        \label{lst:tflite:sinewave:settings}
    \end{listing}
\end{frame}
\subsection{Globals}
\begin{frame}
    \begin{listing}[H]
        \inputsource[fontsize=\fontsize{9}{9}, firstline=22, lastline=35]{c}{arduino/sinewave-test.c}
        \caption{Project code: global variables.}
        \label{lst:arduino:tflite:sinewave:globals}
    \end{listing}
\end{frame}
\subsection{Setup}
\subsubsection{Logging}
\begin{frame}
    \begin{listing}[H]
        \inputsource[firstline=38, lastline=48]{c}{arduino/sinewave-test.c}
        \caption{Project code: error logging.}
        \label{lst:arduino:tflite:sinewave:setup:logging}
    \end{listing}
\end{frame}
\subsubsection{Model}
\begin{frame}
    \begin{listing}[H]
        \inputsource[fontsize=\fontsize{10}{10}, firstline=50, lastline=63]{c}{arduino/sinewave-test.c}
        \caption{Project code: load the model.}
        \label{lst:arduino:tflite:sinewave:setup:model}
    \end{listing}
\end{frame}
\subsubsection{Interpreter}
\begin{frame}
    \begin{listing}[H]
        \inputsource[fontsize=\fontsize{9}{10}, firstline=65, lastline=80]{c}{arduino/sinewave-test.c}
        \caption{Project code: create an interpreter.}
        \label{lst:arduino:tflite:sinewave:setup:interpreter}
    \end{listing}
\end{frame}
\subsubsection{Connect}
\begin{frame}
    \begin{listing}[H]
        \inputsource[fontsize=\fontsize{7}{8}, firstline=82, lastline=99]{c}{arduino/sinewave-test.c}
        \caption{Project code: connect input and output.}
        \label{lst:arduino:tflite:sinewave:setup:connect}
    \end{listing}
\end{frame}
\subsection{Loop}
\begin{frame}
    \begin{listing}[H]
        \inputsource[fontsize=\fontsize{7}{8}, firstline=102]{c}{arduino/sinewave-test.c}
        \caption{Project code: infer sinus value from fixed input.}
        \label{lst:arduino:tflite:sinewave:loop}
    \end{listing}
\end{frame}

\begin{frame}
    \begin{figure}
        \includegraphics[width=0.85\textwidth]{images/microcontroller/sinewave-serial-plotter.png}
        \caption{Serial plotter.}
    \end{figure}
\end{frame}
\subsection{Control}
\begin{frame}
    \begin{listing}[H]
        \inputsource[firstline=13, lastline=20, highlightlines=18]{c}{arduino/sinewave.c}
        \caption{Project code: reduce frequency for better visibility.}
        \label{lst:arduino:tflite:sinewave:control:settings}
    \end{listing}
\end{frame}

\begin{frame}
    \begin{listing}[H]
        \inputsource[gobble=2, fontsize=\fontsize{10}{10}, firstline=107, lastline=131, highlightlines={107,111,126}]{c}{arduino/sinewave.c}
        \caption{Project code: infer sinus value from continuous input values.}
        \label{lst:arduino:tflite:sinewave:control:loop}
    \end{listing}
\end{frame}

\appendix

\begin{frame}[allowframebreaks]{Acronyms}
    \glsadd{ml}
    \printglossary[type=\acronymtype, nonumberlist]
\end{frame}

% \begin{frame}[label=references]{References}
%     \bibliography{references}
% \end{frame}

\end{document}